\documentclass[a4paper,11pt]{article}

%\usepackage{fullpage}
\usepackage{hyperref}
\usepackage{fixltx2e}
\usepackage{amsmath}
\usepackage{graphicx}
\usepackage{fancyhdr}
\usepackage{listings}



\begin{document}

\title{Natuurlijke taalmodellen en interfaces}
\author{Eszter Fodor \& Sharon Gieske}
\date{\today}
\maketitle

\section*{Question 1}
\textit{grep '\^{}A:'} Deze command zoekt naar een regel die met de empty string (\^{}) begint gevolgd door de letter \textit{A} en dubbelepunt.\\\\
\textit{sed 's/\^{}A: //':} Deze command verwijdert de empty string gevolgd door de letter A en een dubbelepunt \textit{(\^{}A:)} van de door grep gevonden regels. De command \textit{$>$ ovis-trainset.txt} definieert de output file.

\section*{Question 2}
\textit{grep -v '\^{}\$':} Deze command zoekt naar niet lege regels. \textit{-v} betekent dat er juist \textit{niet} naar de gespecificeerde regels moet worden gezocht, in dit geval een lege regel (\^{}\$).\\
\textit{sed 's/\textbackslash s \textbackslash + / \textbackslash n/g':} Verwijdert aan het eind van alle regels de spatie (\textbackslash s) en maakt er een newline van. \\
\textit{uniq -c:} Haalt duplicates uit de input en telt (\textit{-c}) hoeveel keer iedere regel voorkomt. \\
\textit{sort -g -r -k 1:} Sorteert de resultaten aan de hand van de hoeveelheid voorkomens en print deze in reversed volgorde (van het hoogste getal naar het laagste). \textit{-g}: sorteer aan de hand van getallen, \textit{-r}: reverse de volgorde van sorteren, \textit{-k 1}: sorteer op de eerste key.\\

\section*{Question3}
Zipf distribution. Dat het een logaritmische functie is.

\newpage

\section*{Question 4}
Met de volgende opeenvolgende commands kan de word-frequency van austen.txt geplot worden: 
\begin{lstlisting}[frame=single]
less austen.txt | sed 's/\s\+/\n/g' | grep -v '^$' 
	| sort | uniq -c | sort -g -r -k 1 
	> austen-freq.txt
	
gnuplot
gnuplot> plot 'austen-freq.txt' 
	using (log($0)):(log($1))
	gnuplot>set terminal jpeg
gnuplot> set output 'austen.jpg'
gnuplot> plot 'austen-freq.txt' using 
	(log($0)):(log($1))
\end{lstlisting}


\noindent Als output plaatje krijgen we dan:\\
\includegraphics[scale=0.4]{austen.jpg}

\noindent De twee plaatjes van de word-frequencies lijken vrijwel identiek.
\newpage

\section*{Question 5}
\textbf{10 most frequent bigrams with their frequencies: }\\
1487, ik wil\\
514, dank u\\
399, nee dank\\
390, wil van\\
279, den haag\\
264, tien uur\\
249, u wel\\
223, acht uur\\
212, wil om\\
205, negen uur\\

\noindent \textbf{10 most frequent trigrams with their frequencies: }\\
393, nee dank u\\
387, ik wil van\\
248, dank u wel\\
211, ik wil om\\
164, nee ik wil\\
153, ik wil graag\\
141, om tien uur\\
128, van den haag\\
117, ja dat klopt\\
117, om acht uur\\
117, ja dat klopt\\
117, om acht uur\\

\noindent \textbf{10 most frequent tetragrams with their frequencies: }\\
177, nee dank u wel\\
81, ik wil reizen van\\
63, ik wil graag van\\
53, van den haag centraal\\
50, den haag centraal naar\\
46, nee hoor dank u\\
46, nee dat hoeft niet\\
46, nee hoor dank u\\
46, nee dat hoeft niet\\
45, van den haag naar\\
42, dat is niet nodig\\
41, ik wil van den\\



\end{document}